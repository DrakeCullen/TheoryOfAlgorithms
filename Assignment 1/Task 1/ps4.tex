\documentclass[12pt,twoside]{article}

\usepackage{amsmath}
\newcommand{\profs}{Professors Srini Devadas and Erik Demaine}
\newcommand{\subj}{6.006}

\newlength{\toppush}
\setlength{\toppush}{2\headheight}
\addtolength{\toppush}{\headsep}

\newcommand{\htitle}[3]{\noindent\vspace*{-\toppush}\newline\parbox{6.5in}
{\textit{Introduction to Algorithms: 6.006}\hfill\newline
Massachusetts Institute of Technology \hfill #3\newline
\profs\hfill Handout #1\vspace*{-.5ex}\newline
\mbox{}\hrulefill\mbox{}}\vspace*{1ex}\mbox{}\newline
\begin{center}{\Large\bf #2}\end{center}}

\newcommand{\handout}[3]{\thispagestyle{empty}
 \markboth{Handout #1: #2}{Handout #1: #2}
 \pagestyle{myheadings}\htitle{#1}{#2}{#3}}

\setlength{\oddsidemargin}{0pt}
\setlength{\evensidemargin}{0pt}
\setlength{\textwidth}{6.5in}
\setlength{\topmargin}{0in}
\setlength{\textheight}{8.5in}

\begin{document}


\handout{8}{Problem Set 4}{March 29, 2008}
\setlength{\parindent}{0pt}

\newcommand{\solution}{
  \medskip
  {\bf Solution:}
}

This problem set is due {\bf Thursday April 10} at {\bf 11:59PM}. 

Solutions should be turned in through the course website in PDF form
using \LaTeX\ or scanned handwritten solutions. 

A template for writing up solutions in \LaTeX\ is available on the
course website.

Remember, your goal is to communicate. Full credit will be given only
to the correct solution which is described clearly. Convoluted and
obtuse descriptions might receive low marks, even when they are
correct. Also, aim for concise solutions, as it will save you time
spent on write-ups, and also help you conceptualize the key idea of
the problem.

\medskip

\hrulefill

\medskip

Exercises are for extra practice and should not be turned in.

{\bf Exercises:}

\begin{itemize}

\item CLRS 22.2-3 (page 539)

\item CLRS 22.2-8 (page 539)

\item CLRS 22.3-9 (page 548)

\item CLRS 22.3-10 (page 549)

\end{itemize}

\hrulefill

\begin{enumerate}

\item {\bf (7 points)} Connected Components

  An undirected graph can be separated into connected components.
  A \emph{connected component} is a set of vertices for which
  (1)~every two vertices in the set are connected by a path, and
  (2)~no edge connects a vertex inside the set to a vertex outside the set.

  Give an $O(V+E)$-time algorithm for dividing an undirected graph into
  connected components, that is, for finding all of the connected components
  in the graph.

\item {\bf (28 points)} Cycle Detection

  A cycle is a path of edges from a node to itself.

  \begin{enumerate}

  \item {\bf (7 points)} You are given a directed graph $G=(V, E)$,
    and a special vertex $v$. Give an algorithm based on BFS that
    determines in $O(V+E)$ time whether $v$ is part of a cycle.
  
  \item {\bf (14 points)} You are given a graph $G=(V, E)$, and you
    are told that every vertex is reachable from vertex~$s$.
    You want to determine whether the graph has any cycles.
    Ben Bitdiddle proposes the following algorithm.
    Perform a BFS from~$s$.  If, during the search, you ever reach
    a node that you have already seen before, then declare that $G$
    has a cycle. If you never reach the same node twice, declare
    that there is no cycle.

    \begin{enumerate}

    \item Show that Ben's algorithm works for undirected graphs.
      
    \item Show that Ben's algorithm does not work for directed graphs.
      
    \end{enumerate}
  
  \item {\bf (7 points)} You are given a directed graph
    $G=(V,E)$. Give an algorithm based on DFS that determines in
    $O(V+E)$ time whether there is a cycle in $G$.
  

  \end{enumerate}

\item {\bf (25 points)} $2 \times 2 \times 2$ Rubik's Cube

  We say that a configuration of the cube is $k$ levels from the
  solved position if you can reach the position in exactly $k$ twists
  (quarter-turns, either clockwise or counterclockwise), but you
  cannot reach the position in fewer twists.

  Download \texttt{ps4\_rubik.zip} from the class website.

  \begin{enumerate}

  \item {\bf (10 points)} Use breadth-first search to recreate the
    chart seen on \\
    \texttt{http://en.wikipedia.org/wiki/Pocket\_Cube}.  Write a
    function \\ \texttt{positions\_at\_level} in \texttt{level.py}
    that takes an argument \texttt{level}, a nonnegative integer.
    Your function should return the number of configurations of the
    cube \texttt{level} levels from the solved position
    (\texttt{rubik.I}), using the \texttt{rubik.quarter\_twists} move
    set.

    Test your code using \texttt{test\_level.py}, and submit it to the
    class website. Testcases above level 10 are commented out, because
    they may require at least 1GB of RAM. Level 10 should take no more
    than a couple minutes, even with 512MB of RAM.


  \item {\bf (15 points)} Now you will actually solve a given configuration
    of the cube, by finding the shortest path between two configurations
    of the cube (the start and the goal).

    Your code from part (a) could easily be modified to find shortest
    paths, but a BFS that goes as deep as 11 levels takes a few minutes
    (not to mention the memory needed). A few minutes might be fine
    for creating a Wikipedia page, but we want to solve the cube fast!

    Instead, we will take advantage of a property of the graph that we
    can see in the chart. In particular, the number of nodes at level
    7 (half the diameter) is much smaller than half the total number
    of nodes. 
    
    With this in mind, we can instead do a two-way BFS, starting from
    each end at the same time, and meeting in the middle. At each
    step, expand one level from the start position, and one level from
    the end position, always checking to see whether any new nodes have
    been discovered in both searches. When you find such a node,
    you just have to read off parent pointers to return the correct path. 

    Write a function \texttt{shortest\_path} in \texttt{solver.py} that
    takes two positions, and returns a list of moves that is a
    shortest path between the two positions.

    Test your code using \texttt{test\_solver.py}. Check that your
    code runs at close to the same speed as level 7 from part (a) in
    the worst case.

  \item {\bf (Optional)} Test your code using
  \texttt{test\_human\_solver.py}, which will ask you to input the
  current configuration of a real Rubik's cube, and then tell you the
  shortest path in human-readable symbols (read \texttt{rubik.py} to
  understand these symbols).

  \end{enumerate}

\end{enumerate}

\end{document}
