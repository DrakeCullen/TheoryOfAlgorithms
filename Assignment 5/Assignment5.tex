\documentclass[11pt]{article}
\usepackage{geometry}                % See geometry.pdf to learn the layout options. There are lots.
\geometry{letterpaper}                   % ... or a4paper or a5paper or ... 
%\geometry{landscape}                % Activate for for rotated page geometry
%\usepackage[parfill]{parskip}    % Activate to begin paragraphs with an empty line rather than an indent
\usepackage{graphicx}
\usepackage{color}
\usepackage{amssymb}
\usepackage{epstopdf}
\usepackage{amsmath}
\usepackage{sectsty}
\usepackage{algorithm}
\usepackage{algorithmic}
\usepackage[hyphens]{url}  %% be sure to specify the option 'hyphens'
\DeclareGraphicsRule{.tif}{png}{.png}{`convert #1 `dirname #1`/`basename #1 .tif`.png}

\title{Drake-Cullen-Assignment5}
\author{Drake Cullen}
%\date{}                                           % Activate to display a given date or no date

\newcommand\tab[1][1cm]{\hspace*{#1}}

\begin{document}
%\maketitle
%\section{}
%\subsection{}
\begin{minipage}{\linewidth}% to keep image and caption on one page
\centering
\includegraphics[keepaspectratio=true,scale=0.35]{CMU.png}
\end{minipage}
\section*{ \centering Assignment 5}
\subsection*{ \centering By Drake Cullen} 

\vspace{5mm}
 
I declare that all material in this assessment task is my work except where there is clear acknowledgement or reference to the work of others. I further declare that I have complied and agreed to the CMU Academic Integrity Policy at the University website. http://www.coloradomesa.edu/student-services/documents
\begin{center}
\textbf{Author’s Name:} Drake Cullen 
\textbf{UID(700\#):} 700480375
\textbf{ Date:} 11/23/2021
\end{center} 

\section{Chapter 3, Question 1)}
Topological sorting is an ordering of vertices in the manner that  for every directed edge u v, the vertex u comes before v in the ordering. In figure 3.10, there are a total of 6 unique topological orderings. We know that f will always be the last node, a is always the first node, c relies on b, and e relies on d. Here are the 6 topological orderings:

\begin{align*}
a, b, c, d, e, f \\
a, b, d, c, e, f\\
a, b, d, e, c, f\\
a, d, e, b, c, f\\
a, d, b, e, c, f\\
a, d, b, c, e, f \\
\end{align*}

\section{Chapter 3, Question 2)}
In class we learned that a graph contains a cycle if the graph contains a backward edge. The DFS algorithm can be run on the graph to perform edge classification. A backward edge will appear if there is a connection from a node to its ancestor.


\begin{algorithm}[H]
\caption{DFS\_Cycle(graph, vertex, visited, parent)}
\begin{algorithmic} 
\STATE // return true if a cycle (backward edge) is found
\STATE visited[vertex] = true
\FOR{neighbor in adj[vertex]}
\IF{neighbor not in visited}
\IF{DFS\_Cycle(graph, neighbor, visited, vertex}
\RETURN true
\ENDIF
\ELSE
\IF{neighbor not equal to parent}
\RETURN true
\ENDIF
\ENDIF
\ENDFOR
\RETURN false
\end{algorithmic}
\end{algorithm}

\section{Chapter 3, Question 3)}



\end{document}  