\documentclass[11pt]{article}
\usepackage{geometry}                % See geometry.pdf to learn the layout options. There are lots.
\geometry{letterpaper}                   % ... or a4paper or a5paper or ... 
%\geometry{landscape}                % Activate for for rotated page geometry
%\usepackage[parfill]{parskip}    % Activate to begin paragraphs with an empty line rather than an indent
\usepackage{graphicx}
\usepackage{color}
\usepackage{amssymb}
\usepackage{epstopdf}
\usepackage{sectsty}
\usepackage[hyphens]{url}  %% be sure to specify the option 'hyphens'
\DeclareGraphicsRule{.tif}{png}{.png}{`convert #1 `dirname #1`/`basename #1 .tif`.png}

\title{Drake-Cullen-Part-A-Answers}
\author{Drake Cullen}
%\date{}                                           % Activate to display a given date or no date

\begin{document}
%\maketitle
%\section{}
%\subsection{}
\begin{minipage}{\linewidth}% to keep image and caption on one page
\centering
\includegraphics[keepaspectratio=true,scale=0.35]{CMU.png}
\end{minipage}
\section*{ \centering Part A Answers}
\subsection*{ \centering By Drake Cullen} 

\vspace{5mm}
 
I declare that all material in this assessment task is my work except where there is clear acknowledgement or reference to the work of others. I further declare that I have complied and agreed to the CMU Academic Integrity Policy at the University website. http://www.coloradomesa.edu/student-services/documents
\begin{center}
\textbf{Author’s Name:} Drake Cullen 
\textbf{UID(700\#):} 700480375
\textbf{ Date:} 10/17/2021
\end{center} 

\section{Statement a)}
The statement is \textbf{True}. Start by finding the weight with the minimum value (smallest negative value). Add a constant factor large enough to make that value positive, and add that value to every other weight. Now, every weight is a positive value. We know that Dijkstras algorithm can be run on graphs with all positive values.

\newpage
\section{Statement b)}
The statement is \textbf{False}. Weights with a value of one cause a problem. Other positive integers will grow larger when squared, but one will keep the same value when squared; therefore, the weights won't increase proportionally. The statement would be true if every weight was multiplied by a constant such as 2. The manner in which the statement fails can be seen in the example below. 

\end{document}  